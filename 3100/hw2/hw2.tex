\documentclass[letterpaper,11pt]{article}
\usepackage{titlesec}
\usepackage{enumitem}
\usepackage[empty]{fullpage}
\usepackage{graphicx}
\usepackage{float}
\usepackage{amsfonts} 
\usepackage{amsmath}
\graphicspath{ {../} }

\begin{document}
\section{Using properties of probability laws:}
\begin{enumerate}[label=\alph*.]
  \item \begin{math}
    Pr(C \cup P) = Pr((C' \cap P')' = 40\%
  \end{math}
  \item \begin{math}
    Pr(C \cap P) = Pr(C) + Pr(P) - Pr(C \cup P) =
    20\% + 30 \% - 40 \% = 10\%
  \end{math}
\end{enumerate}

\section{Probability calculation:}
\begin{enumerate}[label=\alph*.]
  \item \begin{math}
    \Omega = \{1,2,3,4,5,6\} \newline
    Pr(\textrm{outcome is less than four}) = Pr(1) + Pr(2) +
    Pr(3), \newline
    Pr(odds) + Pr(evens) = 1 \newline
     2*Pr(odds) = Pr(evens) \newline
     Pr(odds) = \frac{1}{3} \newline
     Pr(evens) = \frac{2}{3} \newline
     Pr(\textrm{outcome is less than four}) = \frac{1}{9} + \frac{2}{9} + \frac{1}{9} = 
     \frac{4}{9}
  \end{math}
  \item \begin{math}
    Pr(\textrm{odds}) = \frac{1}{3}
  \end{math}
  ,work shown above.
\end{enumerate}
\section{Manhattan revisited:}
\begin{math}
  A = \{(x,y)\bigg||x| + |y| \leq a, \; 0 \leq x \leq 1, \; 0 \leq y \leq 1\, \; 0 \leq a \leq 2\}
  \newline Pr(A) = \frac{1}{2}a\textsuperscript{2} \; \textrm{for } 0 \leq a \leq 1 \; \textrm{and }
  a-\frac{1}{2}(2-a)\textsuperscript{2} \; \textrm{for } 1 < a \leq 2
\end{math}
\section{Partition of a sample space:}
\begin{enumerate}[label=\alph*.]
  \item \[     \Omega = \bigcup_{i=1}^nS_i \]
  \begin{displaymath}
    Pr(A) =  \sum_{i=1}^n Pr(A \cap S_i) = 
    Pr\bigcup_{i=1}^n(A \cap S_i)=
    Pr((A\cap S_1) \cup (A\cap S_2) \cup (A \cap S_3) ...) = 
  \end{displaymath}
  \begin{displaymath}
    Pr(A\cap (S_1 \cup S_2 \cup S_3 ...)) = Pr(A\cap \Omega) = Pr(A)
  \end{displaymath}
  \item Make a disjoint partition: \begin{math}
    \Omega = B \cup C \cup (A \cap B' \cap C') - A \cap B \; \textrm{(subtract off double counted elements)}
  \end{math}
  \newline From this partition, apply result from part a.
\end{enumerate}
\section{Drawing names from a hat:}
\begin{enumerate}[label=\alph*.]
  \item There is only one way that every student draws their own name because the order matters.
  Since the order matters, there are n! total ways to draw. There are n possibilities
  the first time and for each n possibility there are n-1 possibilities for the next draw
  and so on. As a result, the probability is given by $\frac{1}{n!}$
  \item There is only one way that the first m students will draw their name.
  For that one way, there are (n-m-1)! ways the remaining students can draw a name.
  As a result, the probability is given by $\frac{(n-m-1)!}{n!}$
  \item There are m! ways that everyone among the first m students to draw gets a name of
  the last m students to draw. For these m! ways, there are (n-m-1)! ways the remaining students can
  draw a name. As a result, the probability is given by $\frac{(n-m-1)!m!}{n!}$
  \section{ECE faculty needs more exercise:}
  \begin{enumerate}[label=\alph*.]
    \item There are ${16 \choose 6} = 8008$ ways to form rosters.
    \item  There are ${5 \choose 3}$ ways to choose 2 women,
    and for each of these ways there are ${11 \choose 4}$ ways to choose 4 men.
    ${5 \choose 2} * {11 \choose 4} = 10 * 330 = 3300$ ways to have teams with 
    two women and four men.
    \item Sum the ways to choose 2 women 4 men, 3 women 3 men, 4 women 2 men, and 5 women 1 man.
    ${5 \choose 2} * {11 \choose 4} + {5 \choose 3} * {11 \choose 3} + 
    {5 \choose 4} * {11 \choose 2} + {5 \choose 5} * {11 \choose 1} = 10 * 330 + 
    10 * 165 + 5 * 55 + 1 * 11 = 5236$ ways to have teams with at least two women.
    \item i. Probability randomly selected team has exactly two women = $\frac{3300}{8008}$
    \newline ii. Probability randomly selected team has at least two women = $\frac{5236}{8008}$
  \end{enumerate}
\end{enumerate}
\end{document}