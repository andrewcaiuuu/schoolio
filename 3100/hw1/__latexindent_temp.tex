\documentclass[letterpaper,11pt]{article}
\usepackage{titlesec}
\usepackage{enumitem}
\usepackage[empty]{fullpage}
\usepackage{graphicx}
\usepackage{float}
\graphicspath{ {C:/Users/caian/Pictures/3100hw1/} }

\begin{document}

\section{Taking ECE 3100 until A+}
\begin{enumerate}[label=\alph*.]
  \item A good sample space is the set of numbers
  $\Omega = \{1, 2, 3, 4, 5, 6 ...\}$ where the number
  represents that amount of times the student took ECE 3100
  to get an A+. 

  \item $n$ is the event that you get an A+ by the $n$th try.
  \item $\infty$ is the event that you continue taking 
  ECE 3100 forever. 
\end{enumerate}
\section{Constructing a sample space}
\begin{enumerate}[label=\alph*.]
  \item A good sample space is the unit square,
  $\Omega = \{(x,y) | 0 \leq x \leq 1, 0 \leq y \leq 1 \}$
  \item The event is a portion of the unit square, \newline
  $\Omega = \{(x,y)||x|+|y| \leq a, 0 \leq x \leq 1, 0 \leq y \leq 1\}$
\end{enumerate}
\section{Events and set operations}
\begin{itemize}
  \item[] $A = \{2,4,6\} \; B=\{4,5,6\}$ \newline
  \begin{math}
    (A \cup B)\textsuperscript{c} = A \textsuperscript{c}
    \cap B \textsuperscript{c} = \{1,3,5\} \cap \{1,2,3\} = 
    \{1,3\} \newline
    (A \cap B)\textsuperscript{c} = A \textsuperscript{c}
    \cup B \textsuperscript{c} = \{1,3,5\} \cap \{1,2,3\} = 
    \{1,2,3,5\}
  \end{math} 
\end{itemize}
\section{Events and set operations}
\begin{enumerate}[label=\alph*.]
  \item \begin{math}
    A \textsuperscript{c} = (A\textsuperscript{c} \cap B)
    \cup (A\textsuperscript{c} \cap B \textsuperscript{c}) =
    A \textsuperscript{c} \cap (B \cup B \textsuperscript{c}) = 
    A \textsuperscript{c} \cap \Omega = A \textsuperscript{c}
    \newline
    B \textsuperscript{c} = (A \cap B \textsuperscript{c})
    \cup (A \textsuperscript{c} \cap B \textsuperscript{c}) =
    (A \cup A \textsuperscript{c}) \cap B \textsuperscript{c} =
    \Omega \cap B\textsuperscript{c} = B \textsuperscript{c}
  \end{math}
  \item \begin{math}
    (A \cap B)\textsuperscript{c} = (A\textsuperscript{c} 
    \cap B) \cup (A\textsuperscript{c} \cap B\textsuperscript{c})
    \cup (A \cap B\textsuperscript{c}) = (A 
    \cup B\textsuperscript{c})\textsuperscript{c} \cup (A \cup B)\textsuperscript{c}
    \cup (A\textsuperscript{c} \cup B)\textsuperscript{c} = 
    \newline ((A \cup B\textsuperscript{c}) \cap (A \cup B))\textsuperscript{c} 
    \cup (A\textsuperscript{c} \cup B)\textsuperscript{c} =
    (A \cap (B\textsuperscript{c} \cup B))\textsuperscript{c} \cup (A\textsuperscript{C} \cup B)\textsuperscript{c}
    = (A \cap \Omega)\textsuperscript{c} \cup (A\textsuperscript{c}\cup B)\textsuperscript{c} =
    \newline A\textsuperscript{c} \cup (A\textsuperscript{c} \cup B)\textsuperscript{c} = 
    (A \cap (A\textsuperscript{c}\cup B))\textsuperscript{c} =
    ((A \cap A\textsuperscript{c}) \cup (A \cap B))\textsuperscript{c} = 
    (\emptyset \cup (A \cap B))\textsuperscript{c} = 
    (A\cap B)\textsuperscript{c}
  \end{math}
  \item \begin{math}
    A = \{1,3,5\} \; B = \{1,2,3\},
    \newline (\{1,3,5\}\cap \{1,2,3\})\textsuperscript{c} =
    (\{1,3\})\textsuperscript{c} = \{2,4,5,6\} = 
    (\{2,4,6\} \cap \{1,2,3\}) \cup (\{2,4,6\} \cap \{4,5,6\})
    \cup (\{1,3,5\} \cap \{4,5,6\}) = \{2\} \cup \{4,6\} \cup \{5\} =
    \{2,4,5,6\}
  \end{math}
\end{enumerate}
\section{Events and Venn Diagram}
\begin{enumerate}[label=\alph*.]
  \item 
    \begin{figure}[H]
    \includegraphics{a}
  \end{figure}
  $A \cap B \cap C\textsuperscript{c}$
  \item  
  \begin{figure}[H]
  \includegraphics{b}
  \end{figure}
  $A\cup B\cup C$
  \item   \begin{figure}[H]
    \includegraphics{c}
    \end{figure}
  $(A\cup B\cup C) \cap (A\cap B\cap C)\textsuperscript{c} 
  \cap (A\cap B)\textsuperscript{c} \cap 
  (A\cap C)\textsuperscript{c} \cap 
  (B\cap C)\textsuperscript{c}$
  \item \begin{figure}[H]
    \includegraphics{d}
    \end{figure}
  $(A\cup B\cup C)\textsuperscript{c}$
  \item \begin{figure}[H]
    \includegraphics{e}
    \end{figure}
  $A \cap B \cap C$
\end{enumerate}

\end{document}